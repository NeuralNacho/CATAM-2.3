\documentclass[12pt, a4paper]{article}

\usepackage[medium]{titlesec}
\usepackage{graphicx}
\usepackage{caption}
\usepackage{amsmath}
\usepackage{amsthm}
\usepackage[legalpaper, portrait, margin=1.2in]{geometry}
\usepackage{wrapfig}
\usepackage{varwidth}
\usepackage{listings}
\usepackage{amssymb}
\usepackage{enumitem}
\usepackage{bm}

\allowdisplaybreaks  % will allow page break in align environment

\begin{document}
	
\setlength{\parindent}{0pt}
\captionsetup{justification=centering}
\lstset{
	showstringspaces=false
}


\begin{titlepage}
	\Huge
	\textbf{2.3}
	\begin{center}
		\vspace*{7cm}
		
		\Huge
		\textbf{Additional Project \\ Curves in the Complex Plane}
		
		\vspace{0.5cm}
	\end{center}
\end{titlepage}

\section{Complex roots of $f(z)$}

\subsubsection*{Programming Task: Plot $f(C_{r})$ and find closest point to origin}

A program for this task can be found on page \pageref{programmingtask1}. 
Below is sample output for $r = 2$, demonstrating that the program works as intended.

\vspace{0.4cm}
\begin{minipage}{\textwidth}
	\centering
	\includegraphics[width=0.75\linewidth]{ProgTask1_results}
	\vspace{0.4cm}\\
	\includegraphics[width=0.75\linewidth]{ProgTask1_fig}
	\captionof{figure}{Sample output for $r = 2$}
\end{minipage}

%In the program, 100,000 points on the circle $C_{r}$, with the angle increasing linearly, are mapped to generate the graph. We show that the modulus is calculated to a high degree of accuracy using this mapping.\\
%Let $w_{0}=f_{1}(z_{0})$ be the closest point to $0+0i$.\\
%$z_{0} = r\text{exp}(i(\frac{2n\pi}{100000}+\epsilon))$ where $|\epsilon|\leq\frac{\pi}{100000}$.\\
%Let $w = f_{1}(r\text{exp}(i(\frac{2n\pi}{100000}))$. Then the modulus of the closest calculated point, r, satisfies $r\leq|w|$.
%|w-w_{0}| \leq ... |\leq 0.001(r^3+r^2+sqrt(5)r)|
% So accuracy of smallest modulus (r) is < 0.001 (for r near 1) using this many points
	

\subsubsection*{Question 1}

We know there are 3 roots since $f_{1}$ is a cubic. Plotting a graph of the modulus of the closest point against $r$ for $r$ between $0$ and $2$, it is clear that the roots are near 0.6, 1.4 and 1.6. 
\vspace{0.2cm}\\
\begin{minipage}{\textwidth}
	\includegraphics[width=\linewidth]{q1_fig1}
	\captionof{figure}{Modulus of closest point to $0+0i$ for $r\in[0,2]$}
\end{minipage}
\vspace{0.4cm}

The function $find$\textunderscore$decimal$ on page \pageref{finddec} is then used to work out the optimal value of r for an increasing number of decimal places. Using $0.6$ as an example, we perform calculations as follows:
\begin{itemize}
	\item[--] Find minimum of the modulus of the closest point on $f(C_{r})$ to $0+0i$ for \\ $ r= 0.50, 0.51, ... 0.69, 0.70$.
	\item[--] Find that $0.62$ is minimal so we know that to 2d.p. the value of $r$ at the root is $0.61, 0.62$ or $0.63$ since the moduli of the closest points for different $r$ decreases as $r$ approaches the root value.
	\item[--] Find minimum from $0.610, 0.611, ... 0.629, 0.630$.
	\item[--] After the 4th recursion we obtain the result $r = 0.61803$ so it must be that $r = 0.6180$ to 4d.p.
\end{itemize}

Then doing the same for 1.4 and 1.6, we obtain $r = 1.4142$ and $r = 1.6180$ also to 4d.p. \vspace{0.2cm}% We also output the value of $z$ (to 3s.f.) used to obtain this $r$ and, letting $z = re^{i\theta}$, we can find $\theta$ for these values of $z$.
%\begin{align}  % confusing to have this here
%	&r = 0.6180, ~ z = 0.618i, ~~~~~~~~ \theta\approx\pi/2  \nonumber\\
%	&r = 1.4142, ~ z = 1.00 + 1.00i, ~ \theta\approx\pi/4 \nonumber\\
%	&r = 1.6180, ~ z = -1.62i, ~~~~~~~~ \theta\approx3\pi/2 \nonumber
%\end{align}

Let $z$ be the value we approximate the root to be and write  $z = re^{i\theta}$. We need to ensure that the real and imaginary parts of $z$ are accurate up to 3s.f. From the $find$\textunderscore$decimal$ function, we also output the values of $z$ at the optimal values of $r$ (however we cannot be sure they are accurate to 3s.f.) and find that $\theta$ is approximately $1.57, 0.79, 4.71$ for $r = 0.6180, 1.4142, 1.6180$ respectively. \vspace{0.2cm} \\
% Of course we know in reality these are pi/2, pi/4, 3pi/2 respectively
Let the exact values of $r$ and $\theta$ at the root be $r_{0}$ and $\theta_{0}$ respectively. For the three roots, we know $r_0$ to 4d.p. so we can say $r_{0}\in[r-5\times10^{-5}, r+5\times10^{-5}]$.
\vspace{0.2cm}\\
Then we can use a very small step size for $\theta$ of $2\pi/350,000$ on a small arc of $C_{r}$ (for fast calculation) around the values of $\theta$ above to find the optimal $z$ with more accuracy. We obtain $\theta = 1.5708..., 0.7854..., 4.7123...$ and these results are accurate to within $0.00001$ radians since $\pi/350,000<10^{-5}$.
%Then we repeat the $find$\textunderscore$decimal$ algorithm with a very small step size for $\theta$ of $2\pi/350,000$ on a small arc of $C_{r}$ around the values of $\theta$ above and obtain the same values of $z$ as above.  Then $\theta$ is accurate to within $\pi/350,000<0.00001$ radians.
% Note couldn't use this angle before in find_decimal since then generating 350,000 points and doing that like 4x20 times so millions of calculations. But now we've narrowed down region can focus on small arc
\begin{align}
	\therefore \text{cos}(\theta_{0})&\in[\text{cos}(\theta-10^{-5}), \text{cos}(\theta+10^{-5})] \nonumber\\
	&\subset[{\text{cos}(\theta)-10^{-5}, \text{cos}(\theta)+10^{-5}}]\nonumber\\
	\implies(r-5\times10^{-5})(\text{cos}(\theta)&-10^{-5})\leq r_{0}\text{cos}(\theta_{0})\leq (r+5\times10^{-5})(\text{cos}(\theta)+10^{-5}) \nonumber\\
	r<2 &\text{ and } |\text{cos}(\theta)|<1,  \nonumber\\
	\implies  r\text{cos}(\theta)&-7\times10^{-5} < r_{0}\text{cos}(\theta_{0}) < r\text{cos}(\theta)+7\times10^{-5} \nonumber
%	\implies \text{The }& \text{real part of $z$ is accurate to 3d.p.}\nonumber
\end{align}
The same applies to  $r\text{sin}(\theta)$. For $r = 0.6180$, we have,
\begin{align*}
	& r\text{cos}(\theta) + ir\text{sin}(\theta) = 0.61803i \\
	\implies & r_{0}\text{cos}(\theta_{0}) \in [-7\times10^{-5}, 7\times10^{-5}] \\
	& r_{0}\text{sin}(\theta_{0}) \in [0.61803-7\times10^{-5}, 0.61803+7\times10^{-5}] = [0.61796, 0.6181] \\
	\implies & z = 0.618i \text{~ is accurate to 3s.f. for ~} r = 0.618.
\end{align*}

% Calculating $z = r\text{cos}(\theta)+r\text{sin}(\theta)$, we find,
%\begin{align}  % This is not nice
%	&r = 0.6180, ~ z \in [0.61803i - 0.00007(1+i),0.61803i + 0.00007(1+i)] \text{ ~~~(2D box)} \nonumber\\
%	&r = 1.4142, ~ z \in [1.000000 + 1.00000i - 0.00007(1+i), 1.000000 + 1.00000i + 0.00007(1+i)] \nonumber\\
%	&r = 1.6180, ~ z \in [-1.61803i - 0.00007(1+i), -1.61803i + 0.00007(1+i)] \nonumber
%\end{align}
Doing the same calculations for the other $r$, it follows that to 3s.f. the values of $z$ for the corresponding $r$ are,
\begin{align}
	& z =  0.618i && z = 1.00 + 1.00i &&& z = -1.62i\nonumber\\
	& r = 0.618 && r = 1.414, &&& r = 1.618 ~~ \nonumber 
\end{align}


\subsubsection*{Question 2}

\begin{align}
	f'_{1}(z) = 3z^{2}-2z+2-i\nonumber
\end{align}

We follow the same procedure as in Question 1, first plotting a graph to find that the values of $r$ at the roots are approximately 0.8 and 0.95. Inputting these values into the recursive function we obtain,
\begin{align}
	&r = 0.785, ~ z \approx 0.12-0.78i, ~ \theta\approx4.86\nonumber\\ % r = 0.7847 to 4d.p.
	&r = 0.950, ~ z \approx 0.55+0.78i, ~ \theta\approx0.96\nonumber % r = 0.9498 to 4d.p.
\end{align}
Then, again using a step size for $\theta$ of $2\pi/350,000$ on a small arc of $C_{r}$, we increase the accuracy of the result for $r$ to 5d.p. and find values of $z$. Using the same idea as in Question 1, this time the real and imaginary parts are accurate to $1.2\times10^{-5}$ (not $7\times10^{-5}$ like in Question 1 since $r$ now accurate to 1 more d.p.). We find:
% 0.00007 not sufficient for 0.11847 since 0.11847+0.00007 = 0.11854 rounds to 0.119 which is incorrect
\begin{align*}  % confusing to have this here
	& r = 0.78470, ~ \text{Re}(z) \in [0.11847 - 1.2\times10^{-5}, 0.11847 + 1.2\times10^{-5}] = [0.118458, 0.118482] \\
	& ~~~~~~~~~ \text{Im}(z) \in [-0.77571 - 1.2\times10^{-5}, -0.77571 + 1.2\times10^{-5}] = [-0.775722, -0.775698] \\
	\\
	& r = 0.94986, ~ ~ \text{Re}(z) \in [0.54819 - 1.2\times10^{-5}, 0.54819 + 1.2\times10^{-5}] = [0.548178, 0.548202] \\
	& ~~~~~~~~~~~~~~~~~~~ \text{Im}(z) \in [0.77570 - 1.2\times10^{-5}, 0.77570 + 1.2\times10^{-5}] = [0.775688, 0.775712]
	% z \in [0.11847 - 0.77571i - 0.000012(1+i),0.11847 - 0.77571i + 0.000012(1+i)] \\
	% &r = 0.94986, ~ z \in [0.54819 + 0.77570i - 0.000012(1+i), 0.54819 + 0.77570i + 0.000012(1+i)]
\end{align*}
It follows that to 3s.f. the values of $z$ for the corresponding $r$ must be,
\begin{align}
	& z = 0.118 - 0.776i && z = 0.548 + 0.776i \nonumber\\
	& r = 0.785 && r = 0.950 \nonumber 
\end{align}

\section{Images $f(C_{r})$ of $C_{r}$}


\subsubsection*{Question 3}

The change made to the program is simply to add the function to be plotted as a parameter in the $generate$\textunderscore$points$ function found in the program for programming task 1 on page \pageref{programmingtask1}.
\vspace{0.1cm}\\
\begin{minipage}{\textwidth}
	\includegraphics[width=\linewidth]{q3_fig1}
	\captionof{figure}{Plots of $g(C_{r})$ for different values of $r$}
	\label{q3_fig1}
\end{minipage}
\renewcommand\labelitemi{{\boldmath$\cdot$}} % change bullet point size
\begin{itemize}[topsep = 8pt, leftmargin = *]
	\itemsep 0em
	\item As $r$ increases, the size of the shape formed by $g(C_{r})$ also increases since the magnitude of the $z$ and $z^{3}$ terms increases.
	\item For very small $r$, such as $r=0.1$ in Figure \ref{q3_fig1}, $g(C_{r})$ approximates a circle. This is because $z$ is much larger than $z^{3}$ here so the curve is almost the circle from the image of $z$. 
%	\item  For large $r$ the curve resembles a circle as demonstrated by $r=5$ in figure \ref{q3_fig}. This is because the $z^3$ term is much larger than the $z$ term here so the curve is almost the circle from the image of $z^{3}$.
	\item As $r$ increases up to $1$/$\sqrt3$, the top and bottom of this circle begin curving inwards. Then two loops form simultaneously, one at the top and another at the bottom of the curve $g(C_{r})$.
	\item The loops appear for $r>1$/$\sqrt3$. This is derived by taking $z=re^{i\theta}$ and finding the derivative of the real part of $g$ w.r.t. $\theta$ to be $-3r^{3}sin(3\theta)+rsin(\theta)$ with $r=1$/$\sqrt3$ as the positive root when $\theta=\pi/2$ ($g(z)$ is pure imaginary for this $\theta$).
	\item An intuitive explanation for the loops is that we have a circle from $z^3$ which loops three times added to a circle from $z$ which loops once. Then once the $z^{3}$ term is large enough to have an effect the loops start developing.
% May want to revisit intuitive explanation after Q7 k_tot
	\item We can examine the curve for $\theta=0$ to $\theta=\pi/2$ and then use symmetry to form the rest of the curve.\vspace{0.2cm}\\
	\begin{minipage}{\textwidth}
		\centering
		\includegraphics[width=0.75\linewidth]{q3_fig2}
		\captionof{figure}{Formation of $g(C_{0.75})$ from a quarter of $C_{0.75}$}
		\label{q3_fig2}
	\end{minipage}
	\vspace{0.2cm}\\
	The black lines in Figure \ref{q3_fig2} show points which are added to create points in the subplot to the right of Figure \ref{q3_fig2}. We can think of the orange curve distorting the green curve to create the blue curve. Using the reflection symmetry in the real and imaginary axes, we obtain the curve for $r = 0.75$ in Figure \ref{q3_fig1}. 
	\item The loops grow in size as $r$ increases and overlap for $r>1$. For large $r$ the curve again resembles a circle as demonstrated by $r=5$ in figure \ref{q3_fig1}.
\end{itemize}


\subsubsection*{Question 4}

\begin{minipage}{\textwidth}
	\includegraphics[width=\linewidth]{q4_fig1}
	\captionof{figure}{Plots of $h(C_{r})$ for different values of $r$}
	\label{q4_fig1}
\end{minipage}
\begin{itemize}[topsep = 8pt, leftmargin = *]
	\itemsep 0em
	\item The change in $h(C_{r})$ as $r$ increases is similar to in Question 3 since two loops develop and because for very small and large $r$, the curve resembles a circle. This is clear to see in Figure \ref{q4_fig1}.
	\item The curves are shifted to the right of the origin because of the $+1$ term in $h(z)$.
	\item Unlike in Question 3, the loops develop to the left and right of the curve as the terms of $h$ pulls the curve inward on the real axis rather than on the imaginary axis.
%	\item The looping can be thought of as being caused by the same kind of distortion described in Question 3. 
	\item For $r < 1$ a right loop develops before the left loop. For these smaller values of $r$ the $z^3$ is relatively small so can be ignored. % Plotting without z^3 term confirms this to be true 
	Then the $-z^{2}$ term (whose image is a circle looping twice) stretches points near the real axis on the circle formed by $1-z$ to the left. % Also points on top and bottom stretched right
	This results in right side of $h(C_{r})$ being more warped than the left side. The image below represents this visually. \vspace{0.2cm}\\
	\begin{minipage}{\textwidth}
		\centering
		\includegraphics[width=0.5\linewidth]{q4_fig2}
%		\captionof{figure}{}
		\label{q4_fig2}
	\end{minipage}
	\item The left loop emerges at the same time as the the two sides of the curve intercept at $0+0i$ as shown in Figure \ref{q4_fig1} using $r=1$ and $r=1.05$. This is because $h(z)=(z-1)^{2}(z+1)$ so there is a triple root for $r=1$ which is only possible if we have the configuration described.
	\item For larger $r$ there is a transition where the curve has 2 points of self-intersection rather than 4. This transition happens at around $r=2.2$ and is because for larger $r$ the $z^3$ term dominates $h$ causing the former right loop to grow such that it encircles the entire curve. 
% Don't start with loops inside loops
\end{itemize}


\subsubsection*{Question 5}

\begin{minipage}{\textwidth}
	\includegraphics[width=\linewidth]{q5_fig1}
	\captionof{figure}{Plots of $f_{1}(C_{r})$ for different values of $r$}
	\label{q5_fig1}
\end{minipage}
\begin{itemize}[topsep = 8pt, leftmargin = *]
	\itemsep 0em
	\item Similar to the curves $g(C_{r})$ and $h(C_{r})$, two loops develop as $r$ increases. In this case, they're positioned down and to the right which can be explained as follows: without the $-z^{2}$ term, the curves $f_{1}(C_{r})$ are like the plots in Question 3 Figure \ref{q3_fig1} but rotated by $arg(2-i)$ clockwise. Then $-z^{2}$ shifts these the loops to the right and shifts the lower right section of the curve inwards.
	\item The lower loop develops before the loop to the right, as shown by the curve for $r = 0.9$ in Figure \ref{q5_fig1}. The same ideas as Question 4 help explain this; for smaller $r$ we get one loop developing faster as we the $z^{3}$ term has limited effect. Here this corresponds to the lower loop developing faster.
	% (2-i)z is circle. But iz rotates circle! So theta = 0 not corresponding to point on real axis
	\item As the loops grow they intersect and overlap in the same way as the curves in Questions 3 and 4. Then for very large $r$, the curve has a circular shape.
\end{itemize}


\section{Curvature of images $f(C_{r})$ of $C_{r}$}


\subsubsection*{Question 6}

\begin{align}
	|\kappa| & = \left\vert\frac{d\bm{t}(s)}{d\phi}\frac{d\phi}{ds}\right\vert\nonumber\\
	\bm{t}  & = \frac{d\bm{x}}{d\phi}\text{\bigg/}\left\vert\frac{d\bm{x}}{d\phi}\right\vert \text{ ~~\&~~ } \left\vert\frac{ds}{d\phi}\right\vert = \left\vert\frac{d\bm{x}}{d\phi}\right\vert = |\bm{x'}|. ~~ \text{Therefore,}\nonumber\\
	|\kappa| & = \left\vert \frac{d}{d\phi}\left(\frac{\bm{x'}}{|\bm{x'}|}\right) \cdot |\bm{x'}|^{-1}\right\vert \nonumber\\
	& = \left\vert\frac{|\bm{x'}|\bm{x''} - \bm{x'}\frac{d}{d\phi}|\bm{x'}|}{|\bm{x'}|^{3}}\right\vert \nonumber\\
	\frac{d}{d\phi} & |\bm{x'}|^{2} = 2\bm{x'}\cdot\bm{x''} \text{ ~~\&~~ } \frac{d}{d\phi}|\bm{x'}|^{2} = 2|\bm{x'}|\frac{d}{d\phi}|\bm{x'}| \implies \frac{d}{d\phi}|\bm{x'}| = \left\vert \frac{\bm{x'} \cdot \bm{x''}}{\bm{x'}} \right\vert \nonumber\\
	\therefore ~ |\kappa| & = \left\vert\frac{|\bm{x'}|^{2}\bm{x''} - \left\vert \bm{x'}\cdot\bm{x''} \right\vert\bm{x'}}{|\bm{x'}|^{4}}\right\vert \nonumber\\
	& = \left\vert \frac{|\bm{x'} \times (\bm{x''} \times \bm{x'})}{|\bm{x'}|^{4}} \right\vert \text{ ~using triple product identity} \nonumber\\
	~ \bm{x''} & \text{ and } \bm{x''} \times \bm{x'} \text{ orthogonal} \implies |\kappa| = \frac{|\bm{x'} \times \bm{x''}|}{|\bm{x'}|^{3}} ~~~ \qed \nonumber
\end{align}

Finding expressions for $\bm{x'}$ and $\bm{x''}$:
\begin{align}
	\bm{t}(s) = & \frac{\bm{x'}}{|\bm{x'}|} \text{ ~~\&~~ } |\bm{x'}| = \left\vert\frac{ds}{d\phi}\right\vert \implies \bm{x'} = \left\vert\frac{ds}{d\phi}\right\vert \bm{t}(s) \nonumber\\
	\bm{k}(s) & = \frac{d\bm{t}(s)}{ds}  = \frac{d}{d\phi}\left({\left\vert\frac{ds}{d\phi}\right\vert^{-1} \bm{x'}}\right)  \left(\frac{ds}{d\phi}\right)^{-1} \nonumber\\
	\frac{d}{d\phi}\left( \left\vert\frac{ds}{d\phi}\right\vert \right)^{2} = & ~ 2\frac{ds}{d\phi}\frac{d^{2}s}{d\phi^{2}} \text{ ~~\&~~ } \frac{d}{d\phi}\left( \left\vert\frac{ds}{d\phi}\right\vert \right)^{2} = 2\left\vert\frac{ds}{d\phi}\right\vert \frac{d}{d\phi}\left\vert\frac{ds}{d\phi}\right\vert \nonumber\\
	& \implies \frac{d}{d\phi}\left\vert\frac{ds}{d\phi}\right\vert = \frac{ds}{d\phi}\frac{d^{2}s}{d\phi^{2}} \bigg/ \left\vert\frac{ds}{d\phi}\right\vert \nonumber\\
	\therefore ~ \bm{k}(s) & = \frac{\left(\frac{ds}{d\phi}\right)^{2}\bm{x''} - \frac{ds}{d\phi} \frac{d^{2}s}{d\phi^{2}}\bm{x'}}{\left(\frac{ds}{d\phi}\right)^{3} \left\vert\frac{ds}{d\phi}\right\vert} \nonumber\\
	\therefore ~~~ \bm{x''} & = \left\vert\frac{ds}{d\phi}\right\vert \left(\frac{ds}{d\phi} \bm{k}(s) + \left(\frac{ds}{d\phi}\right)^{-1}  \frac{d^{2}s}{d\phi^{2}} \bm{t}(s)\right) \nonumber
\end{align}

$\bm{k}(s)$ is orthogonal to $\bm{t}(s)$ since $\frac{d}{ds}(\bm{t}\cdot\bm{t}) = 0 \implies \bm{t}\cdot\bm{k} = 0$. Then  $\bm{t}(s)\times\bm{k}(s)$ is orthogonal to the plane the curve is in. $\bm{k}(s)$ points towards the local centre of rotation. Therefore, using the right hand rule, the curve is turning anticlockwise at $\bm{x}(s)$ iff  the component of $\bm{t}(s)\times\bm{k}(s)$ orthogonal to the plane is $> 0$. Hence sgn$(\kappa)$ = sgn$(\bm{t}(s)\times\bm{k}(s))$, where sgn of the vector is taken to mean the sign of the component orthogonal to the plane.
\begin{align}
	\bm{x'}\times\bm{x''} = \left\vert\frac{ds}{d\phi}\right\vert^{2} \frac{ds}{d\phi} \left(\bm{t}(s)\times\bm{k}(s)\right) \nonumber\\
	\implies \text{sgn}(\kappa) = \text{sgn}\left(\frac{\bm{x'}\times\bm{x''}}{ds/d\phi}\right)	\nonumber
\end{align}

In our case, $\frac{ds}{d\phi} > 0$ since distance along the curve increases as $\phi$ increases, hence,
$\text{sgn}(\kappa) = \text{sgn}\left(\bm{x'}\times\bm{x''}\right)$.
\\

Writing (8) and (9) in terms of derivatives of $f(z)$ with respect to $z$:
\begin{align}
	& \frac{df(z(\phi))}{d\phi} = \frac{df(z)}{dz}\frac{dz}{d\phi} \nonumber\\
	& \frac{d^{2}f(z(\phi))}{d\phi^{2}} = \frac{df(z)}{dz} \frac{d^{2}z}{d\phi^{2}} + \frac{d^{2}f(z)}{dz^{2}} \left(\frac{dz}{d\phi}\right)^{2} \nonumber\\
	& z = re^{i\phi} \text{ along } C_{r} \implies \frac{dz}{d\phi} = ire^{i\phi} = iz, ~~ \frac{d^{2}z}{d\phi^{2}} = -re^{i\phi} = -z \nonumber\\
	& \therefore ~ \frac{df(z(\phi))}{d\phi} = ~ i\frac{df(z)}{dz}re^{i\phi} \nonumber\\
	& \frac{d^{2}f(z(\phi))}{d\phi^{2}} = -re^{i\phi}\left(\frac{df(z)}{dz} + \frac{d^{2}f(z)}{dz^{2}}re^{i\phi}\right) \nonumber\\
	\therefore ~ \bm{x'} = & \left(\text{Re}\left[i\frac{df(z)}{dz}z\right], \text{Im}\left[i\frac{df(z)}{dz}z\right] \right) ~~~~~~~~~~~~~~~~~~~~~~~~~~~~~~~~~~~~~~~~~~~~~~~~~~~~~~ \text{(8)} \nonumber\\
	\bm{x''} = & \left(\text{Re}\left[-z\frac{d}{dz}\left(\frac{df(z)}{dz}z\right) \right], \text{Im}\left[-z\frac{d}{dz}\left(\frac{df(z)}{dz}z\right) \right] \right) ~~~~~~~~~~~~~~~~~~~~~~~~~~~~~~~ \text{(9)} \nonumber
	% OR \bm{x''} = & \left(\text{Re}\left[-z\left(\frac{df(z(\phi))}{dz} + \frac{d^{2}f(z(\phi))}{dz^{2}}z\right) \right], \text{Im}\left[ -z\left(\frac{df(z(\phi))}{dz} + \frac{d^{2}f(z(\phi))}{dz^{2}}z \right) \right] \right) ~~~~~ \text{(9)} \nonumber
\end{align}

\subsubsection*{Programming Task: Write a program to compute the integral $\kappa_{tot}$}

\begin{align}
	\kappa_{tot} = \int_{0}^{2\pi} \kappa |\bm{x'}| d\phi \nonumber
\end{align}
The integral is calculated using this equation, and the trapezium rule. The code for this program can be found on page \pageref{programmingtask2}.

\subsubsection*{Question 7}

\begin{minipage}{\textwidth}
	\includegraphics[width=\linewidth]{q7_fig1}
	\captionof{figure}{Plots of $\kappa_{tot}$ against $r$ for $f_{1}(z)$, $g(z)$ and $h(z)$}
	\label{q7_fig1}
	% not too many points plotted because when loop first develop derivative almost zero
\end{minipage}\vspace{0.4cm}

From the graphs, it is clear to see that the value of $k_{tot}$ jumps and plateaus at increasing multiples of $2\pi$. The jumps are at the values of $r$ where there is a root in the derivative of the function for this $r$. For instance, the two values of $r$ at the roots found in Question 2 correspond exactly with the jumps in Figure \ref{q7_fig1} for $f_{1}(z)$. Additionally, there is a jump at $r = 1/\sqrt{3}$ in $g(z)$ where we said the loop starts to form for $g(C_{r})$. This demonstrates that the curvature jumps when another (anticlockwise) loop forms and this is precisely where we will find roots in the derivative.   Numerically, when a new small loop forms, the value of $|\bm{x'}|$ on this loop will also be small, so we get a spike in the value of $\kappa$ using the formula in Question 6. % ??? USe Q6 
\\\\
For a general polynomial $f(z)$, $\kappa_{tot}$ tells us the number of $2\pi$ rotations anticlockwise someone would experience as they move along the curve. This is supported by all the evidence of the previous paragraph.
% it is the turning number


\pagebreak
\textbf{programming\textunderscore task\textunderscore1.py for programming task}\centering\label{programmingtask1}
\lstinputlisting[language=Python]{C:/Users/angus/OneDrive/Documents/CATAM 2.3 Code/CATAM-2.3/programming_task_1.py}
\vspace{2cm}

\pagebreak
\textbf{q1and2\textunderscore functions.py used in questions 1 and 2}\centering\label{finddec}
\lstinputlisting[language=Python]{C:/Users/angus/OneDrive/Documents/CATAM 2.3 Code/CATAM-2.3/q1and2_functions.py}
\vspace{2cm}

\pagebreak
\textbf{question\textunderscore 1.py for question 1}\centering
\lstinputlisting[language=Python]{C:/Users/angus/OneDrive/Documents/CATAM 2.3 Code/CATAM-2.3/question_1.py}
\vspace{6cm}

\pagebreak
\textbf{question\textunderscore 2.py for question 2}\centering
\lstinputlisting[language=Python]{C:/Users/angus/OneDrive/Documents/CATAM 2.3 Code/CATAM-2.3/question_2.py}
\vspace{2cm}

\pagebreak
\textbf{question\textunderscore 3.py for question 3}\centering
\lstinputlisting[language=Python]{C:/Users/angus/OneDrive/Documents/CATAM 2.3 Code/CATAM-2.3/question_3.py}
\vspace{6cm}

\textbf{question\textunderscore 4.py for question 4}\centering
\lstinputlisting[language=Python]{C:/Users/angus/OneDrive/Documents/CATAM 2.3 Code/CATAM-2.3/question_4.py}
\vspace{2cm}

\pagebreak
\textbf{question\textunderscore 5.py for question 5}\centering
\lstinputlisting[language=Python]{C:/Users/angus/OneDrive/Documents/CATAM 2.3 Code/CATAM-2.3/question_5.py}
\vspace{2cm}

\pagebreak
\textbf{programming\textunderscore task\textunderscore 2.py for programming task}\centering\label{programmingtask2}
\lstinputlisting[language=Python]{C:/Users/angus/OneDrive/Documents/CATAM 2.3 Code/CATAM-2.3/programming_task_2.py}
\vspace{2cm}

\pagebreak
\textbf{question\textunderscore 7.py for question 7}\centering
\lstinputlisting[language=Python]{C:/Users/angus/OneDrive/Documents/CATAM 2.3 Code/CATAM-2.3/question_7.py}
\vspace{2cm}

\end{document}